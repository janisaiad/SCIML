\documentclass[aspectratio=169]{beamer}
\usepackage{listings}
\usepackage{xcolor}

\definecolor{codegreen}{rgb}{0,0.6,0}
\definecolor{codegray}{rgb}{0.5,0.5,0.5}
\definecolor{codepurple}{rgb}{0.58,0,0.82}
\definecolor{backcolour}{rgb}{0.95,0.95,0.92}

\lstdefinestyle{mystyle}{
    backgroundcolor=\color{backcolour},   
    commentstyle=\color{codegreen},
    keywordstyle=\color{magenta},
    numberstyle=\tiny\color{codegray},
    stringstyle=\color{codepurple},
    basicstyle=\ttfamily\scriptsize,
    breakatwhitespace=false,         
    breaklines=true,                 
    captionpos=b,                    
    keepspaces=true,                 
    numbers=left,                    
    numbersep=5pt,                  
    showspaces=false,                
    showstringspaces=false,
    showtabs=false,                  
    tabsize=2
}

\lstset{style=mystyle}

\begin{document}

\begin{frame}[fragile]
\frametitle{Fourier Layer Call Implementation}

\begin{lstlisting}[language=Python]
def call(self, inputs: tf.Tensor) -> tf.Tensor:
    with tf.device(self.device):
        casted_data = tf.cast(inputs, tf.complex64) # [batch, p_2,p_2,1]
        function_fft = tf.signal.fft2d(casted_data) # [batch, n_modes, n_modes
        fourier_casted = tf.cast(self.fourier_weights, 
                               tf.complex64) # [batch, n_modes, dim_coords]

        function_fft = function_fft * fourier_casted # [batch, p_2, p_2, 1]
        x_spatial = tf.signal.ifft2d(function_fft) # [batch, p_2, p_2, 1]
        x_spatial = tf.cast(x_spatial, tf.float32) # [batch, p_2, p_2, 1]
        z = self.linear_layer(inputs) # [batch, p_2, p_2, 1]
        return self.activation(x_spatial + z) # [batch, p_2, p_2, 1]
\end{lstlisting}

\end{frame}

\end{document}
