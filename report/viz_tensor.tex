\documentclass[tikz,border=10pt]{standalone}

%% Language and font encodings
\usepackage[english]{babel}
\usepackage[utf8x]{inputenc}
\usepackage[T1]{fontenc}
\usepackage{tikz}
\usepackage{xcolor}
\usepackage{amsmath}
\usepackage{tikz-3dplot}

% Définition des couleurs
\definecolor{encodercolor}{RGB}{70,130,180}   % Bleu acier
\definecolor{fourierlayercolor}{RGB}{220,20,60}  % Rouge cramoisi
\definecolor{decodercolor}{RGB}{46,139,87}    % Vert mer
\definecolor{inputcolor}{RGB}{255,165,0}      % Orange
\definecolor{outputcolor}{RGB}{148,0,211}     % Violet
\definecolor{fftcolor}{RGB}{255,105,180}      % Rose vif
\definecolor{spectralcolor}{RGB}{100,149,237} % Bleu cornflower
\definecolor{weightcolor}{RGB}{255,215,0}     % Or

\usetikzlibrary{backgrounds,calc,shadings,shapes.arrows,arrows,shapes.symbols,shadows,positioning,decorations.markings,backgrounds,arrows.meta,3d}

% Configuration de la vue 3D
\tdplotsetmaincoords{70}{110}

\begin{document}

% Première visualisation - Flux des tenseurs à travers le réseau FNO
\begin{tikzpicture}[tdplot_main_coords, scale=0.8]
    % Définition des styles
    \tikzset{
        cuboide/.style={
            opacity=0.8,
            draw=black,
            thick,
            fill=#1
        },
        fleche/.style={
            ->,
            >=stealth,
            thick,
            draw=black!70
        },
        etiquette/.style={
            font=\small,
            align=center
        }
    }
    
    % Fonction pour dessiner un cuboïde 3D
    \newcommand{\drawcuboid}[7]{% x, y, z, dx, dy, dz, color
        \begin{scope}[shift={(#1,#2,#3)}]
            % Face avant
            \fill[cuboide=#7] (0,0,0) -- (#4,0,0) -- (#4,#5,0) -- (0,#5,0) -- cycle;
            % Face droite
            \fill[cuboide=#7,opacity=0.6] (#4,0,0) -- (#4,#5,0) -- (#4,#5,#6) -- (#4,0,#6) -- cycle;
            % Face supérieure
            \fill[cuboide=#7,opacity=0.7] (0,#5,0) -- (#4,#5,0) -- (#4,#5,#6) -- (0,#5,#6) -- cycle;
            % Face arrière
            \draw[cuboide=#7,opacity=0.4] (0,0,#6) -- (#4,0,#6) -- (#4,#5,#6) -- (0,#5,#6) -- cycle;
            % Face gauche
            \draw[cuboide=#7,opacity=0.5] (0,0,0) -- (0,#5,0) -- (0,#5,#6) -- (0,0,#6) -- cycle;
            % Face inférieure
            \draw[cuboide=#7,opacity=0.3] (0,0,0) -- (#4,0,0) -- (#4,0,#6) -- (0,0,#6) -- cycle;
            
            % Arêtes
            \draw[black, thick] (0,0,0) -- (#4,0,0) -- (#4,#5,0) -- (0,#5,0) -- cycle;
            \draw[black, thick] (0,0,#6) -- (#4,0,#6) -- (#4,#5,#6) -- (0,#5,#6) -- cycle;
            \draw[black, thick] (0,0,0) -- (0,0,#6);
            \draw[black, thick] (#4,0,0) -- (#4,0,#6);
            \draw[black, thick] (#4,#5,0) -- (#4,#5,#6);
            \draw[black, thick] (0,#5,0) -- (0,#5,#6);
        \end{scope}
    }
    
    % Axes principaux
    \draw[->, thick] (0,0,0) -- (15,0,0) node[right] {$x$};
    \draw[->, thick] (0,0,0) -- (0,10,0) node[above] {$y$};
    \draw[->, thick] (0,0,0) -- (0,0,6) node[above] {$z$ (batch)};
    
    % Tenseur d'entrée [B, p_1, p_1]
    \drawcuboid{0}{0}{0}{3}{3}{2}{inputcolor}
    \node[etiquette, below=0.1cm] at (1.5,1.5,0) {Entrée $[B, p_1, p_1]$};
    \node[etiquette] at (-0.5,0,-0.5) {$p_1=30$};
    \node[etiquette] at (0,-0.5,-0.5) {$p_1=30$};
    \node[etiquette] at (-0.5,-0.5,2) {$B$ (batch)};
    
    % Encodeur
    \node[etiquette, above] at (5,2,2.5) {Encodeur};
    \draw[fleche] (3.5,1.5,1) -- (4.5,1.5,1);
    
    % Sortie de l'encodeur [B, p_1, p_1]
    \drawcuboid{5}{0}{0}{3}{3}{2}{encodercolor}
    \node[etiquette, below=0.1cm] at (6.5,1.5,0) {$[B, p_1, p_1]$};
    
    % FFT
    \draw[fleche] (8.5,1.5,1) -- (9.5,1.5,1);
    \node[etiquette, above] at (9,1.5,1.5) {FFT 2D};
    
    % Domaine de Fourier [B, p_1, p_1] (complexe)
    \drawcuboid{10}{0}{0}{3}{3}{2}{fftcolor}
    \node[etiquette, below=0.1cm] at (11.5,1.5,0) {$[B, p_1, p_1]$ (complexe)};
    
    % Poids spectraux [p_1, p_1]
    \drawcuboid{10}{5}{0}{3}{3}{0.5}{weightcolor}
    \node[etiquette, above=0.1cm] at (11.5,6.5,0.5) {Poids spectraux $[p_1, p_1]$};
    
    % Multiplication
    \draw[fleche] (13.5,1.5,1) -- (14.5,1.5,1);
    \draw[fleche] (11.5,5,0.25) -- (14,2.5,1);
    \node[etiquette] at (14,2,2) {$\times$};
    
    % Résultat dans le domaine spectral
    \drawcuboid{15}{0}{0}{3}{3}{2}{spectralcolor}
    \node[etiquette, below=0.1cm] at (16.5,1.5,0) {$[B, p_1, p_1]$ après mult.};
    
    % IFFT
    \draw[fleche] (18.5,1.5,1) -- (19.5,1.5,1);
    \node[etiquette, above] at (19,1.5,1.5) {IFFT 2D};
    
    % Sortie de la couche de Fourier
    \drawcuboid{20}{0}{0}{3}{3}{2}{fourierlayercolor}
    \node[etiquette, below=0.1cm] at (21.5,1.5,0) {$[B, p_1, p_1]$ après IFFT};
    
    % Chemin linéaire parallèle
    \drawcuboid{5}{5}{0}{3}{3}{2}{inputcolor}
    \node[etiquette, above=0.1cm] at (6.5,6.5,2) {Entrée pour chemin linéaire};
    
    % Couche linéaire
    \draw[fleche] (8.5,6.5,1) -- (14.5,6.5,1);
    \node[etiquette, above] at (11.5,6.5,1.5) {Couche linéaire};
    
    % Sortie du chemin linéaire
    \drawcuboid{15}{5}{0}{3}{3}{2}{decodercolor}
    \node[etiquette, above=0.1cm] at (16.5,6.5,2) {$[B, p_1, p_1]$ linéaire};
    
    % Addition
    \draw[fleche] (18.5,6.5,1) -- (19.5,4,1);
    \draw[fleche] (21.5,1.5,1) -- (19.5,3,1);
    \node[etiquette] at (19.5,3.5,2) {$+$};
    
    % Activation ReLU
    \draw[fleche] (19.5,3.5,1) -- (21,3.5,1);
    \node[etiquette, above] at (21,3.5,1.5) {ReLU};
    
    % Sortie finale d'une couche de Fourier
    \drawcuboid{22}{2}{0}{3}{3}{2}{outputcolor}
    \node[etiquette, below=0.1cm] at (23.5,3.5,0) {$[B, p_1, p_1]$ sortie de couche};
    
    % Titre global
    \node[font=\large\bfseries] at (12.5,10,0) {Traitement des tenseurs dans une couche de Fourier du FNO};
    
    % Légende des dimensions
    \node[draw, rounded corners, fill=white, align=left, text width=10cm] at (13,-2,0) {
        \textbf{Dimensions des tenseurs:}\\
        $B$ = Taille du batch\\
        $p_1$ = Dimension de la grille d'entrée ($30 \times 30$)\\
        Toutes les couches de Fourier préservent les dimensions $[B, p_1, p_1]$
    };
\end{tikzpicture}

% Deuxième visualisation - Focus sur la structure des tenseurs
\begin{tikzpicture}[tdplot_main_coords, scale=1.2]
    % Styles
    \tikzset{
        grid/.style={
            draw=gray!50,
            thin
        },
        cell/.style={
            draw=black,
            fill=#1,
            opacity=0.8
        },
        axes/.style={
            ->,
            >=stealth,
            thick
        }
    }
    
    % Fonction pour dessiner un cuboïde 3D avec grille
    \newcommand{\drawTensorGrid}[7]{% x, y, z, dx, dy, dz, color
        \begin{scope}[shift={(#1,#2,#3)}]
            % Dessiner la structure en grille
            % Face avant
            \foreach \i in {0,...,#4} {
                \draw[grid] (\i,0,0) -- (\i,#5,0);
            }
            \foreach \j in {0,...,#5} {
                \draw[grid] (0,\j,0) -- (#4,\j,0);
            }
            
            % Face droite
            \foreach \i in {0,...,#6} {
                \draw[grid] (#4,0,\i) -- (#4,#5,\i);
            }
            \foreach \j in {0,...,#5} {
                \draw[grid] (#4,\j,0) -- (#4,\j,#6);
            }
            
            % Face supérieure
            \foreach \i in {0,...,#4} {
                \draw[grid] (\i,#5,0) -- (\i,#5,#6);
            }
            \foreach \k in {0,...,#6} {
                \draw[grid] (0,#5,\k) -- (#4,#5,\k);
            }
            
            % Contours extérieurs
            \draw[black, thick] (0,0,0) -- (#4,0,0) -- (#4,#5,0) -- (0,#5,0) -- cycle;
            \draw[black, thick] (0,0,#6) -- (#4,0,#6) -- (#4,#5,#6) -- (0,#5,#6) -- cycle;
            \draw[black, thick] (0,0,0) -- (0,0,#6);
            \draw[black, thick] (#4,0,0) -- (#4,0,#6);
            \draw[black, thick] (#4,#5,0) -- (#4,#5,#6);
            \draw[black, thick] (0,#5,0) -- (0,#5,#6);
            
            % Dessiner quelques cellules pour montrer la structure
            \filldraw[cell=#7] (1,1,0) -- (2,1,0) -- (2,2,0) -- (1,2,0) -- cycle;
            \filldraw[cell=#7] (3,2,0) -- (4,2,0) -- (4,3,0) -- (3,3,0) -- cycle;
            \filldraw[cell=#7] (2,4,0) -- (3,4,0) -- (3,5,0) -- (2,5,0) -- cycle;
            
            \filldraw[cell=#7] (1,1,1) -- (2,1,1) -- (2,2,1) -- (1,2,1) -- cycle;
            \filldraw[cell=#7] (3,3,1) -- (4,3,1) -- (4,4,1) -- (3,4,1) -- cycle;
            
            \filldraw[cell=#7] (2,2,2) -- (3,2,2) -- (3,3,2) -- (2,3,2) -- cycle;
            \filldraw[cell=#7] (4,4,2) -- (5,4,2) -- (5,5,2) -- (4,5,2) -- cycle;
        \end{scope}
    }
    
    % Axes
    \draw[axes] (0,0,0) -- (6.5,0,0) node[right] {axe $i$ (largeur)};
    \draw[axes] (0,0,0) -- (0,6.5,0) node[above] {axe $j$ (hauteur)};
    \draw[axes] (0,0,0) -- (0,0,3.5) node[above] {axe $k$ (batch)};
    
    % Dessiner un tenseur 3D d'entrée
    \drawTensorGrid{0}{0}{0}{5}{5}{2}{inputcolor}
    
    % Étiquettes des dimensions
    \node[align=center] at (2.5,-1,0) {$p_1 = 30$};
    \node[align=center] at (-1,2.5,0) {$p_1 = 30$};
    \node[align=center] at (-1,-1,1.5) {$B$ (batch)};
    
    % Titre
    \node[font=\large\bfseries] at (3,8,0) {Structure d'un tenseur d'entrée dans le FNO};
    
    % Annotations supplémentaires
    \node[draw, rounded corners, fill=white, align=left, text width=10cm] at (10,3,1) {
        \textbf{Structure des tenseurs dans le FNO:}\\
        - Chaque cellule $(i,j)$ représente une valeur sur la grille spatiale\\
        - La dimension $k$ représente les différents exemples du batch\\
        - Les opérations FFT/IFFT sont appliquées sur les dimensions $i,j$\\
        - La dimension de batch $k$ est préservée à travers toutes les couches
    };
    
    % Flèche explicative pour un élément spécifique
    \draw[->, thick, red] (7,3,1) -- (3,3,1);
    \node[align=center, red] at (8,4,1.5) {Élément $(i=3,j=3,k=1)$};
\end{tikzpicture}

% Troisième visualisation - Architecture complète
\begin{tikzpicture}[tdplot_main_coords, scale=0.7]
    % Styles pour les cuboïdes
    \tikzset{
        module/.style={
            draw,
            thick,
            fill=#1,
            opacity=0.8,
            rounded corners
        },
        fleche/.style={
            ->,
            >=stealth,
            thick,
            draw=black
        }
    }
    
    % Fonction pour dessiner un module 3D (parallélépipède pour représenter une couche)
    \newcommand{\drawModule}[7]{% x, y, z, dx, dy, dz, color
        \begin{scope}[shift={(#1,#2,#3)}]
            % Face avant arrondie
            \fill[module=#7] (0,0,0) -- (#4,0,0) -- (#4,#5,0) -- (0,#5,0) -- cycle;
            % Face droite
            \fill[module=#7,opacity=0.6] (#4,0,0) -- (#4,#5,0) -- (#4,#5,#6) -- (#4,0,#6) -- cycle;
            % Face supérieure
            \fill[module=#7,opacity=0.7] (0,#5,0) -- (#4,#5,0) -- (#4,#5,#6) -- (0,#5,#6) -- cycle;
            % Contours
            \draw[black, thick] (0,0,0) -- (#4,0,0) -- (#4,#5,0) -- (0,#5,0) -- cycle;
            \draw[black, thick] (0,0,#6) -- (#4,0,#6) -- (#4,#5,#6) -- (0,#5,#6) -- cycle;
            \draw[black, thick] (0,0,0) -- (0,0,#6);
            \draw[black, thick] (#4,0,0) -- (#4,0,#6);
            \draw[black, thick] (#4,#5,0) -- (#4,#5,#6);
            \draw[black, thick] (0,#5,0) -- (0,#5,#6);
        \end{scope}
    }
    
    % Tenseur d'entrée [B, p_1, p_1]
    \drawcuboid{0}{0}{0}{3}{3}{2}{inputcolor}
    \node[align=center, font=\small] at (1.5,-1,1) {Entrée $[B, p_1, p_1]$};
    
    % Réseau encodeur
    \drawModule{4}{0}{0}{2}{3}{2}{encodercolor}
    \node[align=center, font=\small] at (5,-1,1) {Encodeur};
    \draw[fleche] (3.2,1.5,1) -- (3.8,1.5,1);
    
    % Couches de Fourier
    \drawModule{7}{0}{0}{2}{3}{2}{fourierlayercolor}
    \node[align=center, font=\small] at (8,-1,1) {Fourier Layer 1};
    \draw[fleche] (6.2,1.5,1) -- (6.8,1.5,1);
    
    \drawModule{10}{0}{0}{2}{3}{2}{fourierlayercolor}
    \node[align=center, font=\small] at (11,-1,1) {Fourier Layer 2};
    \draw[fleche] (9.2,1.5,1) -- (9.8,1.5,1);
    
    \drawModule{13}{0}{0}{2}{3}{2}{fourierlayercolor}
    \node[align=center, font=\small] at (14,-1,1) {Fourier Layer 3};
    \draw[fleche] (12.2,1.5,1) -- (12.8,1.5,1);
    
    \drawModule{16}{0}{0}{2}{3}{2}{fourierlayercolor}
    \node[align=center, font=\small] at (17,-1,1) {Fourier Layer 4};
    \draw[fleche] (15.2,1.5,1) -- (15.8,1.5,1);
    
    % Réseau décodeur
    \drawModule{19}{0}{0}{2}{3}{2}{decodercolor}
    \node[align=center, font=\small] at (20,-1,1) {Décodeur};
    \draw[fleche] (18.2,1.5,1) -- (18.8,1.5,1);
    
    % Tenseur de sortie [B, p_3, p_3]
    \drawcuboid{22}{0}{0}{3}{3}{2}{outputcolor}
    \node[align=center, font=\small] at (23.5,-1,1) {Sortie $[B, p_3, p_3]$};
    \draw[fleche] (21.2,1.5,1) -- (21.8,1.5,1);
    
    % Titre global
    \node[font=\large\bfseries] at (12,5,1) {Architecture complète du modèle FNO};
    
    % Légende des dimensions des tenseurs à chaque étape
    \node[draw, rounded corners, fill=white, align=left, text width=14cm] at (12,-4,1) {
        \textbf{Évolution des dimensions des tenseurs:}\\
        Entrée: $[B, p_1, p_1]$ où $p_1 = 30$ (grille spatiale d'entrée)\\
        Encodeur: transforme les données tout en conservant les dimensions $[B, p_1, p_1]$\\
        Couches de Fourier: appliquent des transformations dans le domaine spectral, préservant $[B, p_1, p_1]$\\
        Décodeur: produit la sortie finale de dimensions $[B, p_3, p_3]$ où $p_3 = 30$ (grille de sortie)
    };
\end{tikzpicture}

\end{document} 